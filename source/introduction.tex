\section{Введение}
\subsection*{Цель} \addcontentsline{toc}{subsection}{Цель}
Исследовать устройство электронных ламп диода и триода.

\subsection*{Задачи} \addcontentsline{toc}{subsection}{Задачи}
\begin{enumerate}
	\item \textit{Снятие статической характеристики диода}:
	\begin{enumerate}
		\item Меняя величину $U_a$ для каждого ее конкретного значения считывать величину $I_a$; 
		\item По полученным данным построить соответствующую характеристику и вычислить крутизну в нескольких точках характеристики.
	\end{enumerate}
	\item \textit{Снятие статических характеристик триода}:
	\begin{enumerate}
		\item Начертить график гиперболы $P_{a}^{max} = U_a I_a = const$, чтобы при снятии характеристик рассеиваемая мощность не превышала предельную допустимую для данной лампы;
		\item Снять семейство анодных характеристик при нескольких значениях сеточного напряжения;
		\item Снять семейство сеточных характеристик при различных анодных напряжениях, изменяя $U_c$, чтобы лампа запиралась в одном положении и анодный ток был ниже максимально допустимого в другом;
		\item Для средних участков характеристик вычислить параметры $S$, $R_i$, $\mu$.
	\end{enumerate}
	\item \textit{Определение коэффициента усиления усилителя}:
	\begin{enumerate}
		\item Проверить частотные свойства усилителя. Снять зависимость коэффициента усиления от частоты. Определить частоту, на которой коэффициент усиления уменьшается в 2 раза;
		\item Вычислить $K$ для всех возможных значений сопротивления $R_a$. Проверить, выполняется ли соотношение \cref{eq:task:1}. Если указанное соотношение не выполняется, то объясните, чем это может быть вызвано.
	\end{enumerate}
\end{enumerate}

\subsection*{Приборы и оборудование} \addcontentsline{toc}{subsection}{Приборы и оборудование}

\begin{enumerate}
	\item Источник питания постоянного тока \texttt{Б5-50};
	\item Вольтметр универсальный \texttt{GDM-8245};
	\item Генератор низкочастотных сигналов \texttt{ГЗ-112};
	\item Провода;
	\item Осциллограф цифровой \texttt{GDS-71022};
	\item Установка с диодом и триодом.
\end{enumerate}