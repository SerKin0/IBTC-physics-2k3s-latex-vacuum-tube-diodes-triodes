\subsubsection{Зависимость усиления от сопротивления нагрузки}

Для каждой из характеристик подсчитали коэффициент крутизны $S$ на линейном участке графика.

\begin{table}{H}
	\centering
	\caption{Зависимость $K$ от частоты $\nu$, при $R_a = 50~\text{кОм}$}
	\label{tab:7}
	\begin{tabular}{cc}
			\toprule
			$\nu,~\text{кГц}$ & $2U_\text{вых},~\text{В}$ \\ 
			\midrule
		\end{tabular}
\end{table}

\begin{table}[H]
	\centering
	\caption{Зависимость $K$ от сопротивления нагрузки $R_a$ при частоте $\nu=1~\text{кГц}$}
	\label{tab:7}
	\begin{tabular}{ccc}
			\toprule
			$R,~\text{кОм}$ & $2U_\text{вых},~\text{В}$ & $K$  \\ \midrule
			0,5             & 0,182                     & 0,9  \\
			1,0             & 0,348                     & 1,7  \\
			2,0             & 0,640                     & 3,2  \\
			5,0             & 1,230                     & 6,2  \\
			10,0            & 1,900                     & 9,5  \\
			20,0            & 2,540                     & 12,7 \\
			50,0            & 3,200                     & 16,0 \\
			100,0           & 3,560                     & 17,8 \\
			200,0           & 3,720                     & 18,6 \\
			500,0           & 3,600                     & 18,0 \\
			1000,0          & 3,360                     & 16,8 \\
			\bottomrule
		\end{tabular}
\end{table}

\begin{table}[H]
	\centering
	\caption{Зависимость $K$ от частоты $\nu$ ($U_a = 100~\text{В}$)}
	\label{tab:8}
	\begin{subtable}{0.45\textwidth}
			\centering
			\caption{$R=1~\text{кОм}$}
			\label{tab:8a}
			\begin{tabular}{ccc}
					\toprule
					$\nu,~\text{кГц}$ & $2U_\text{вых},~\text{В}$ & $K$  \\\midrule
					1                 & 0,360                     & 1,80 \\
					20                & 0,352                     & 1,76 \\
					40                & 0,356                     & 1,78 \\
					50                & 0,348                     & 1,74 \\
					70                & 0,352                     & 1,76 \\
					90                & 0,352                     & 1,76 \\
					120               & 0,352                     & 1,76 \\
					140               & 0,352                     & 1,76 \\
					180               & 0,344                     & 1,72 \\
					230               & 0,344                     & 1,72 \\
					270               & 0,336                     & 1,68 \\
					500               & 0,308                     & 1,54 \\
					600               & 0,288                     & 1,44 \\
					700               & 0,272                     & 1,36 \\
					800               & 0,254                     & 1,27 \\
					900               & 0,240                     & 1,20 \\ \bottomrule
				\end{tabular}
		\end{subtable}
	\begin{subtable}{0.45\linewidth}
			\centering
			\caption{$R=50~\text{кОм}$}
			\label{tab:8b}
			\begin{tabular}{ccc}
					\toprule
					$\nu,~\text{кГц}$ & $2U_\text{вых},~\text{В}$ & $K$   \\ \midrule
					1                 & 3,240                     & 16,20 \\
					20                & 3,140                     & 15,70 \\
					40                & 2,840                     & 14,20 \\
					50                & 2,640                     & 13,20 \\
					70                & 2,340                     & 11,70 \\
					90                & 2,000                     & 10,00 \\
					120               & 1,680                     & 8,40  \\
					140               & 1,480                     & 7,40  \\
					180               & 1,200                     & 6,00  \\
					230               & 0,984                     & 4,92  \\
					270               & 0,848                     & 4,24  \\
					500               & 0,464                     & 2,32  \\
					600               & 0,388                     & 1,94  \\
					700               & 0,336                     & 1,68  \\
					800               & 0,288                     & 1,44  \\
					900               & 0,260                     & 1,30 \\
					\bottomrule
				\end{tabular}
		\end{subtable}
\end{table}

Заметим, что при $R = 50~\text{кОм}$ (см \cref{tab:8b}) коэффициент $K$ падает в $2$ раза при частоте  между $\nu = 120~\text{кГц}$ и $\nu = 140~\text{кГц}$

Построим график зависимости коэффициента усиления от сопротивления нагрузки $K(R_a)$ по \cref{tab:7}

%%%
\begin{figure}[H]
	\centering
	\label{fig:5}
	\includegraphics[width=\linewidth]{graph/graph_5}
	\caption{Зависимость коэффициента усиления $K$ от сопротивления $R_a$}
\end{figure}
%\includegraphics[width=\linewidth]{graph/graph_5}
%%%

Погрешности $U_\text{вых}$ мы находим из фотографий измерений с помощью осциллографа. Погрешность $R$ возьмём из документации к установке.

Максимальный коэффициент усиления достигается при $R_a = 200~\text{кОм}$ и равен $18,6$, далее рост усиления прекращается и наблюдается спад усиления. Это соответствует тому, что при увеличении $R_a$ рабочая точка на динамической анодно-сеточной характеристике смещается ближе к основанию, где $R_i$ возрастает, а  уменьшается

Из формулы \cref{eq:link:8} следует, что динамический коэффициент не может превзойти статический. Практически динамический коэффициент оказался меньше статического примерно в два раза.

Построим графики зависимости $K$ от частоты сигнала $\nu$ по \cref{tab:8a} и \cref{tab:8b}.

Из графиков видно, что падение коэффициента усиления, по мере увеличения частоты сигнала, происходит быстрее при более высоком сопротивлении.

 \begin{figure}[H]
 	\centering
 	\label{fig:6}
 	\includegraphics[width=\linewidth]{graph/graph_6}
 	\caption{Зависимость коэффициента усиления от частоты входного сигнала при $R=50~\text{кОм}$}
 \end{figure}
 \begin{figure}[H]
 	\centering
 	\label{fig:7}
 	\includegraphics[width=\linewidth]{graph/graph_7}
 	\caption{Зависимость коэффициента усиления от частоты входного сигнала при $R=1~\text{кОм}$}
 \end{figure}
