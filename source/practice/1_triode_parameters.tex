\subsubsection{Снятие статической характеристики диода}
\begin{figure}[p]
	\centering
	\begin{subfigure}{0.8\textwidth}
		\centering
		\includegraphics[width=\textwidth]{graph/graph_1}
		\caption{ }
		\label{fig:1a}
	\end{subfigure}
	
	\vspace{0.5cm}
	
	\begin{subfigure}{0.8\textwidth}
		\centering
		\includegraphics[width=\textwidth]{graph/graph_1_32}
		\caption{ }
		\label{fig:1b}
	\end{subfigure}
	\caption{Вольт-амперная характеристика диода с погрешностями измерений и теоретической линией}
	\label{fig:1}
\end{figure}
\begin{table}[H]
	\centering
	\caption{Вольт-амперная характеристика (ВАХ) диода и результаты вычислений погрешностей и крутизны}
	\label{tab:1}
	\begin{subtable}{0.3\textwidth}
		\centering
		\caption{ }
		\label{tab:1a}
		\begin{tabular}{cc}
			\toprule
			\(U_a, \text{ В}\) & \(I_a, \text{ мА}\) \\ \midrule
			2          &        3,78         \\
			4          &        10,16        \\
			6          &        16,88        \\
			8          &        24,84        \\
			10         &        32,40        \\
			12         &        40,90        \\
			14         &        49,94        \\
			16         &        62,40        \\
			18         &        73,52        \\
			20         &        86,31        \\ \bottomrule
		\end{tabular}
	\end{subtable}
	\begin{subtable}{0.45\textwidth}
		\centering
		\caption{ }
		\label{tab:1b}
		\begin{tabular}{cccc}
			\toprule
			\(\Delta U_a, \text{ В}\) & \(\Delta I_a, \text{ мА}\) & \(S, \tfrac{\text{мА}}{\text{В}}\) & \(\Delta S, \tfrac{\text{мА}}{\text{В}}\) \\ \midrule
			0,31            &            0,03            &                 --                 &                    --                     \\
			0,32            &            0,04            &                3,19                &                   0,71                    \\
			0,33            &            0,05            &                3,36                &                   0,77                    \\
			0,34            &            0,07            &                3,98                &                   0,94                    \\
			0,35            &            0,08            &                3,78                &                   0,92                    \\
			0,36            &            0,10            &                4,25                &                   1,07                    \\
			0,37            &            0,12            &                4,52                &                   1,17                    \\
			0,38            &            0,14            &                6,23                &                   1,65                    \\
			0,39            &            0,17            &                5,56                &                   1,52                    \\
			0,40            &            0,19            &                6,40                &                   1,79                    \\ \bottomrule
		\end{tabular}
	\end{subtable}
\end{table}

В ходе лабораторного эксперимента проведено снятие статической вольт-амперной характеристики диода. Результаты экспериментальных измерений зависимости анодного тока \(I_a\) от анодного напряжения \(U_a\) представлены в \cref{tab:1a}.

Дифференциальная крутизна вольт-амперной характеристики \(S\) рассчитана в соответствии с уравнением \cref{eq:2} (см. \cref{app:1}) и занесена в \cref{tab:1b}. Проведенный анализ полученных данных демонстрирует нелинейный характер возрастания крутизны характеристики при увеличении анодного напряжения.

Погрешности измерений анодного напряжения \(\Delta U_a\) и анодного тока \(\Delta I_a\) приняты равными приборным погрешностям и вычислены согласно уравнениям \cref{eq:error:1.2,eq:error:2.1} (см. \cref{app:2,app:3}). Погрешность определения крутизны \(\Delta S\) рассчитана по формуле \cref{eq:error:3.2} (см. \cref{app:4}).

Определим среднеарифметическое конечных значений крутизны вольт-амперной характеристики:
\begin{gather} 
	S_\text{ср} = \frac{6,23 + 5,56 + 6,40}{3}~\frac{\text{мА}}{\text{В}} \approx 6,06~\frac{\text{мА}}{\text{В}}\\
	\Delta S_\text{ср} = \frac{1,65 + 1,52 + 1,79}{3}~\frac{\text{мА}}{\text{В}} \approx 1,65~\frac{\text{мА}}{\text{В}}
\end{gather} 
\[S_\text{ср} = (6,06 \pm 1,65)~\frac{\text{мА}}{\text{В}}\]

На \cref{fig:1a} представлен график зависимости \(I_a(U_a)\) с нанесенными экспериментальными точками и областями погрешностей измерений. Теоретическая кривая (пунктир) построена в соответствии с уравнением \cref{eq:52}, из анализа которого определен параметр \(B\) (см. \cref{app:5}):
\[B = (0,97 \pm 0,05)~\frac{\text{мА}}{\text{В}^{3/2}}\]

Закон трёх вторых не применим в области отрицательных и малых положительных (единицы В) анодных напряжений. Из закона следует, что при нулевом напряжении ток анода должен быть равен нулю, а при отрицательном напряжении формула трёх вторых вообще не определена. В реальных диодах при нулевом анодном напряжении уже течёт ненулевой ток электронов от катода к аноду $-$ это явление называется эффектом Эдисона. Полная отсечка тока наступает только тогда, когда анодное напряжение опускается на несколько В ниже нуля. 

Сдвиг характеристики диода влево на $-1,5~\text{В}$ может быть объяснён неэквипотенциальностью катода прямого накала. В 1914 году Уилсон, анализируя ВАХ прямонакальных диодов, предложил уточнённую модель, основанную на формуле Чайлда. В модели Уилсона ток на начальном участке ВАХ пропорционален напряжению в степени $5/2$, а в области средних напряжений ВАХ совпадает с законом трёх вторых. Дополнительный сдвиг характеристики влево на $-0,5~\text{В}$ в рамках модели Чайлда объяснить невозможно. Этот сдвиг — следствие ненулевых начальных скоростей и тепловой диффузии электронов. Ток, текущий сам по себе в диоде с заземлённым анодом — это ток быстрых электронов, способных преодолеть потенциальную яму пространственного заряда. Добавим сдвиг на $-1,5~\text{В}$ в наше предсказание, на рисунке эта кривая представлена сплошной кривой.

Сравнение экспериментальных данных с теоретической зависимостью показывает их удовлетворительное согласование в исследованном диапазоне напряжений, что подтверждает адекватность примененной теоретической модели.