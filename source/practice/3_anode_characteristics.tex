\subsubsection{Снятие анодной характеристики}
Были сняты 4 анодные характеристики для разных напряжений на сетке $U_c$: -4, -2, 0, 2 вольт. Также вычислены их погрешности по формулам  \cref{eq:error:1.2,eq:error:2.1} (см. \cref{app:8,app:9}) и записаны в \cref{tab:4b}.
\begin{table}[h]
	\centering
	\caption{Снятие анодной характеристики триода}
	\label{tab:4}
	\begin{subtable}{0.49\textwidth}
		\centering
		\caption{Результаты измерений}
		\label{tab:4a}
		\begin{tabular}{cccccc}
			\toprule
			$U_a$ & $I_a^{-4\text{В} }$ & $I_a^{ -2\text{В} }$ & $I_a^{ 0\text{В} }$ & $I_a^{ 2\text{В} }$ &  $I_a^{\text{max}}$ \\
			В & мА & мА & мА & мА & мА \\
			\midrule
			10     & --     & --     & 1,39   & 4,67   & 250,00 \\
			20     & --     & --     & 2,35   & 6,01   & 125,00 \\
			30     & --     & 0,16   & 3,39   & 7,50   & 83,33  \\
			40     & --     & 0,54   & 4,50   & 8,97   & 62,50  \\
			50     & --     & 1,13   & 5,77   & 10,47  & 50,00  \\
			60     & 0,12   & 1,90   & 6,97   & 12,00  & 41,67  \\
			70     & 0,34   & 2,81   & 8,33   & 13,59  & 35,71  \\
			80     & 0,72   & 3,82   & 9,72   & 15,20  & 31,25  \\
			90     & 1,26   & 4,92   & 11,19  & 16,88  & 27,78  \\
			100    & 1,95   & 6,18   & 12,73  & 18,57  & 25,00  \\
			110    & 2,77   & 7,46   & 14,32  & 20,31  & 22,73  \\
			120    & 3,72   & 8,77   & 15,45  & --     & 20,83  \\
			130    & 4,75   & 10,16  & 17,60  & --     & 19,23  \\
			140    & 5,93   & 11,65  & --     & --     & 17,86  \\
			150    & 7,15   & --     & --     & --     & 16,67  \\
			160    & 8,44   & --     & --     & --     & 15,63  \\
			170    & 9,84   & --     & --     & --     & 14,71  \\
			180    & 11,27  & --     & --     & --     & 13,89  \\
			\bottomrule
		\end{tabular}
	\end{subtable}
	\hfill
	\begin{subtable}{0.49\textwidth}
		\centering
		\caption{Погрешности данных \cref{tab:4a}}
		\label{tab:4b}
		\begin{tabular}{ccccc}
			\toprule
			$\Delta U_a$ & $\Delta I_a^{-4\text{В} }$ & $\Delta I_a^{ -2\text{В} }$ & $\Delta I_a^{ 0\text{В} }$ & $\Delta I_a^{ 2\text{В} }$ \\
			В & мА & мА & мА & мА \\
			\midrule
			0,3 & -- & -- & 0,02 & 0,03 \\
			0,4 & -- & -- & 0,02 & 0,03\\
			0,4 & -- & 0,02 & 0,03 & 0,04\\
			0,5 & -- & 0,02 & 0,03 & 0,04\\
			0,5 & -- & 0,02 & 0,03 & 0,04\\
			0,6 & 0,02 & 0,02 & 0,03 & 0,04\\
			0,6 & 0,02 & 0,03 & 0,04 & 0,05\\
			0,7 & 0,02 & 0,03 & 0,04 & 0,05\\
			0,7 & 0,02 & 0,03 & 0,04 & 0,05\\
			0,8 & 0,02 & 0,03 & 0,05 & 0,06\\
			0,8 & 0,03 & 0,03 & 0,05 & 0,06\\
			0,9 & 0,03 & 0,04 & 0,05 & --\\
			0,9 & 0,03 & 0,04 & 0,06 & --\\
			1,0 & 0,03 & 0,04 & -- & --\\
			1,0 & 0,03 & -- & -- & --\\
			1,1 & 0,04 & -- & -- & --\\
			1,1 & 0,04 & -- & -- & --\\
			1,2 & 0,04 & -- & -- & --\\
			\bottomrule
		\end{tabular}
	\end{subtable}
\end{table}


\begin{figure}[p]
	\centering
	\includegraphics[width=\textwidth]{graph/graph_3}
	\caption{Анодная характеристика триода}
	\label{fig:3}
\end{figure}
