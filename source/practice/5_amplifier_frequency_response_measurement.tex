\subsubsection{Изучение АЧХ усилителя}

Подав на вход усилителя сигнал амплитудой $U_\text{вх} = 100~\text{В}$ с генератора ГЗ-112, изменяя частоту, сняли зависимость выходной амплитуды от частоты (АЧХ) и рассчитали коэффициент усиления для каждого значения частоты по формуле \cref{eq:pr:1} (см. \cref{app:6}).
\begin{equation} \label{eq:pr:1}
	K = \frac{2 U_\text{вых}}{2 U_\text{вх}} = \frac{2 U_\text{вых}}{2 \cdot 100~\text{мВ}} = \frac{2 U_\text{вых}}{0,2~\text{В}}
\end{equation}

\newpage
\begin{table}[t]
	\centering
	\caption{Снятие амплитудно-частотной характеристики усилителя}
	\label{tab:5}
	\begin{tabular}{ccccc}
		\toprule
		$\nu_\text{уст},~\text{Гц}$ & Множитель & $\nu,~\text{кГц}$ & $2U_\text{вых},~\text{В}$ & $K$ \\
		\midrule
		10 & $10^2$ & 1 & 0,360 & 1,80 \\
		20 & $10^3$ & 20 & 0,352 & 1,76 \\
		40 & $10^3$ & 40 & 0,356 & 1,78 \\
		50 & $10^3$ & 50 & 0,348 & 1,74 \\
		70 & $10^3$ & 70 & 0,352 & 1,76 \\
		90 & $10^3$ & 90 & 0,352 & 1,76 \\
		12 & $10^4$ & 120 & 0,352 & 1,76 \\
		14 & $10^4$ & 140 & 0,352 & 1,76 \\
		18 & $10^4$ & 180 & 0,344 & 1,72 \\
		23 & $10^4$ & 230 & 0,344 & 1,72 \\
		27 & $10^4$ & 270 & 0,336 & 1,68 \\
		50 & $10^4$ & 500 & 0,308 & 1,54 \\
		60 & $10^4$ & 600 & 0,288 & 1,44 \\
		70 & $10^4$ & 700 & 0,272 & 1,36 \\
		80 & $10^4$ & 800 & 0,254 & 1,27 \\
		90 & $10^4$ & 900 & 0,240 & 1,20 \\
		\bottomrule
	\end{tabular}
	\begin{tabular}{c}
		\toprule
		$\Delta \nu,~\text{кГц}$ \\
		\midrule
		0,1 \\
		0,7 \\
		1,1 \\
		1,3 \\
		1,7 \\
		2,1 \\
		5,4 \\
		5,8 \\
		6,6 \\
		7,6 \\
		8,4 \\
		13,0 \\
		15,0 \\
		17,0 \\
		19,0 \\
		21,0 \\
		\bottomrule
	\end{tabular}
\end{table}
\newpage