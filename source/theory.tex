\section{Теоретическая часть}

\subsection{Диод}

Конструктивно диод состоит из баллона (стеклянного, металлического или керамического), в котором создается вакуум $\sim 10^{-7}$ мм ртутного столба, и системы плоских или цилиндрических электродов: катода и анода.

Катод в простейшем случае представляет собой накаливаемую током вольфрамовую нить, которая при достаточно высокой температуре начинает испускать электроны (это явление получило название \textit{термоэлектронной эмиссии}).

Если соединить анод с катодом через чувствительный гальванометр, то можно обнаружить в этой цепи анодный ток, величина определённая электрическая мощность. Электроны, излучаемые катодом, увеличивают свою кинетическую энергию за счет энергии электрического поля. При ударе об анод электрон отдаёт ему свою энергию. Энергия электронов выделяется на аноде в виде тепловой энергии и излучается им в окружающее пространство. С повышением анодного напряжения увеличивается количество достигших анода электронов, их скорость и кинетическая энергия, и, следовательно, возрастает и мощность, рассеиваемая анодом. Для каждого типа ламп существует максимальная допустимая величина этой мощности -- $P_{max}$. Превышение
ее при работе ламп может вывести последнюю из строя. Величину мощности, рассеиваемой анодом, можно подсчитать по формуле
\begin{equation} \label{eq:0}
	P_a = U_a I_a
\end{equation}

Уравнение $P_{a}^{max} = U_a I_a = const$ на плоскости $U_a$, $I_a$ изображается гиперболой, асимптотами которой являются координатные оси. На рис.2 эта кривая отмечена индексом "1". Чтобы лампа не вышла из строя, всегда надо следить за тем, чтобы при работе с ней не превышать максимально допустимой мощности, рассеиваемой анодом, т.е. работать только в области $P_a < P_{a}^{max}$. Для каждого типа ламп величина $P_{a}^{max}$ определяется конструкцией электродов и их геометрическими размерами. У очень мощных ламп для повышения этой мощности приходится применять меры принудительного охлаждения анода (водяное и воздушное охлаждение).

Функция, выражающая зависимость анодного тока диода от величины напряжения накала $I_a = f(U_n)$, называется температурной характеристикой диода. По ней можно судить об эмиссионной способности катода, о возникновении и исчезновении пространственного заряда между катодом и анодом лампы. Для каждого типа ламп существует нормальное напряжение накала, величину которого нельзя превышать.

Представление о работе диода можно составить и не зная его характеристик, если известны параметры диода: крутизна вольт-амперной характеристики 
\begin{equation} \label{eq:1}
	S = \frac{d I_a}{d U_a},
\end{equation}
измеряемая обычно в $\tfrac{\text{мА}}{\text{В}}$ или обратная ей величина -- внутренние сопротивление $R_i = \frac{1}{S}$. Т.к. характеристика диода существенно нелинейна, то это дифференциальные параметры зависят от значений $I_a$ и $U_a$, т.е. $S = f(U_a)$. На практике приближенное значение крутизны можно определить по вольт-амперной характеристике. Взяв небольшое приращение анодного напряжения $\Delta U_a$ так, чтобы в этих пределах участок анодной характеристики можно считать линейным, по характеристике определим приращение тока $\Delta I_a$. И тогда
\begin{equation} \label{eq:2}
	S \approx \frac{\Delta I_a}{\Delta U_a}
\end{equation}

\subsection{Триод}

В триоде помимо катода и анода имеется еще один электрод -- управляющая сетка, расположенная вблизи катода между катодом и анодом. Конструктивно сетка выполняется в виде металлической проволочной спирали. Изменение потенциала сетки по отношению к катоду изменяет величину электрического поля между ними. При потенциале сетки \( U_c > 0 \) (относительно катода) электрическое поле ускоряет электроны \( u \), следовательно, увеличивает \( I_e \); при \( U_c < 0 \) поле между сеткой и катодом оказывает тормозящее действие на электроны \( u \) уменьшает \( I_e \). Таким образом наличие сетки в лампе позволяет управлять величиной анодного тока. Обычно используют тормозящее действие сетки, т.~е. работают при \( U_c < 0 \), так как при этом сеточный ток отсутствует, и в цепи сетки для управления анодным током не затрачивается никакой мощности.

В триоде анодный ток зависит как от напряжения на сетке, так и от напряжения на аноде \( I_e = f(U_c, U_d) \). Эта зависимость (при постоянной температуре катода \( T_h = \mathrm{const} \)) изображается обычно в виде двух семейств статических характеристик (статическими они называются потому, что снимаются на постоянном токе)
\[ I_a = f_1 (U_c)_{U_a = \text{const}} \]
--- статические анодно-сеточные характеристики.
\[ I_a = f_2 (U_a)_{U_c = \text{const}}\]
--- статические анодные характеристики.

При увеличении \( U_a \) анодно-сеточные характеристики смещаются влево. То же происходит с анодными характеристиками при увеличении \( U_c \).

Анодный ток \( I_a \) меняется нелинейно как с изменением \( U_c \), так и с изменением \( U_a \). Параметры 
\begin{align} \label{eq:2.1}
	\begin{aligned}
		S = \frac{\partial I_a}{\partial U_c} = \left[ \frac{d I_a}{d U_c} \right]_{U_a = \text{const}} && \text{и} && R_i = \frac{\partial U_a}{\partial I_a} = \left[ \frac{\partial U_a}{\partial I_a} \right]_{U_c = \text{const}}
	\end{aligned}
\end{align}
по аналогии с диодом называются соответственно крутизной и внутренним сопротивлением триода.

Крутизна характеристики графически может быть определена как тангенс угла наклона касательной к сеточной характеристике в данной точке. Из-за нелинейности сеточной характеристики значение крутизны в каждой точке характеристики различно.

У существующих триодов \( S \) порядка нескольких десятков мА/В. 

Внутреннее сопротивление графически может быть определено как котангенс угла наклона касательной к анодной характеристике в рабочей точке. У триода \( R_i \) принимает значения от нескольких единиц до нескольких десятков кОм. Практически значения \( S \) и \( R_i \) могут быть приближенно определены, как и для диода, через соответствующие приращения.

Параметры \( S \) и \( R_i \) изменяются в зависимости от действующих напряжений на электродах триода. Найдем связь между этими параметрами, вычисляя полный дифференциал функции \( I_a = f (U_c, U_a) \):

\[ dI_a = \frac{\partial I_a}{\partial U_c} dU_c + \frac{\partial I_a}{\partial U_a} dU_a \]

Если \( U_c \) и \( U_a \) изменять так, чтобы анодный ток \( I_a \) оставался постоянным, т.~е. \( dI_a = 0 \), то, учитывая, что изменения сеточного и анодного напряжения при этом различного знака, получим
\[0 = \frac{\partial I_a}{\partial U_c} - \frac{\partial I_a}{\partial U_a} \left[\frac{d U_a}{U_c}\right]_{I_a = \text{const}} \quad \to \quad S= \frac{1}{R_i} \mu \quad \mu = \left( \frac{\partial U_a}{\partial U_c} \right)_{I_a = \text{const}}, \]
где $\mu$ --- статический коэффициент усиления, показывающий во сколько раз действие изменения напряжения на сетке \( U_c \) эффективнее действия такого же изменения напряжения на аноде \( U_a \).

Величина, обратная коэффициенту усиления, которой удобно пользоваться в некоторых случаях, носит название проницаемости лампы и обозначается буквой \( D \): 
\[
D = \frac{1}{\mu}
\]

Таким образом, используя соотношения \( S R_i = \mu \) или \( S R_i D = 1 \), по двум известным параметрам всегда можно найти третий. Эти соотношения справедливы не только для триодов, но и для более сложных электронных ламп, имеющих большое количество электродов.

Из-за того, что сетка гораздо ближе к катоду, чем анод, управляющее действие ее на поток электронов сильнее. Если между сеткой и катодом приложено переменное напряжение, а в анодную цепь лампы включено сопротивление, то анодный ток будет меняться, следуя за изменением напряжения на сетке. Изменение анодного тока приводит к изменению напряжения на сопротивлении в анодной цепи. Таким образом, меняя напряжение на сетке, можно управлять мощностью, выделяемой в анодной цепи, не расходуя никакой энергии в сеточной цепи (если, конечно, мгновенное значение сеточного напряжения всегда отрицательно). Источником энергии при этом является анодная батарея. Это свойство триода позволяет использовать его для усиления и для генерации электрических колебаний.

Упрощенная схема усилительного каскада приведена на рис. 5. Источник \( E_c \) носит название источника сеточного смещения и служит для задания так называемой рабочей точки на анодно-сеточной характеристике (точка \( P \) на рис. 3). Необходимо отметить, что анодно-сеточная характеристика при \( R_a \neq 0 \) будет отличаться от статической анодно-сеточной характеристики, т.к. при изменении \( I_a \) будет меняться напряжение на аноде лампы (\( U_a = E_a - I_a R_a \)). Такая характеристика, называется динамической.

Как и в статическом режиме, управляющее действие сетки при работе лампы с нагрузкой характеризуется крутизной
\[
S_d = \frac{dI_a}{dU_c}
\]

но в отличие от статической эта крутизна определяется при изменяющемся напряжении на аноде. Нетрудно показать, что динамическая крутизна связана со статической соотношением
\[
S_d = \frac{S}{1 + R_a / R_i}
\]

Усиление сигнала, обеспечиваемое лампой, характеризуется динамическим коэффициентом усиления \( K \), равным отношению амплитуды переменного напряжения на сопротивлении нагрузки \( U_{m_a} \) к амплитуде напряжения на сетке \( U_{m_c} \):
\[
K = \frac{U_{m_a}}{U_{m_c}}
\]

Выразим его через статические параметры лампы. Для этой цели запишем коэффициент усиления в виде
\[
K = \frac{dU_a}{dU_c}
\]

заменив амплитуды бесконечно малыми приращениями, что в линейном режиме допустимо, и учитывая также, что \( dU_a = -R_a dI_a \), получим 
\[
K = - \frac{R_a dI_a}{dU_c} = -R_a S_\text{д} = -\frac{R_a S}{1 + R_a / R_i} = \frac{\mu}{1 + R_i / R_a},
\]

Из этой формулы следует, что динамический коэффициент усиления тем больше, чем больше \( R_a \), и \( K \to \mu \) при \( R_a \to \infty \). На практике же при больших значениях \( R_a / R_i \) дальнейшее увеличение \( R_a \) может не дать никакого роста \( K \). Дело в том, что увеличение \( R_a \) при постоянном \( E_a \) приводит к смещению рабочей точки на динамической анодно-сеточной характеристике ближе к излому, где \( R_i \) возрастает, а \( \mu \) уменьшается. Практически в усилителях на триодах \( R_a / R_i \) лежит в пределах 2–5.

С ростом частоты усиливаемого сигнала усилительные свойства триода ухудшаются. Это связано, в основном, с влиянием междуэлектродных емкостей и индуктивностей вводов лампы. Особенно сильное влияние на частотные свойства триода оказывает емкость между сеткой и анодом, которую называют проходной емкостью \( C_{a-c} \). В курсе электроники доказывается, что предельная частота, до которой триод может быть использован как усилитель при заданном коэффициенте усиления \( K \), определяется выражением:

\[
f_{\text{пр}} = 25 \frac{S}{K^2 C_{ac}}, \quad [\text{МГц}]
\]
где \( S \) --– в мА/В, \( C_{a-c} \) --– в пФ.

\subsection{Вывод закона 3/2 Ленгмюра}
Определим зависимость анодного тока от анодного напряжения в диоде, образованном двумя плоскими, безграничными, параллельными друг другу пластинами. В этом случае можно пренебречь краевым эффектом и считать поле между анодом и катодом однородным. Примем следующие допущения. Пусть у поверхности катода $x = 0: U_k = 0$, , начальная скорость электронов $u_0 = 0$.

Если предположить, что электроды вакуумной трубки плоские, а температура катода постоянная, то потенциал электрического поля будет зависеть только от одной координаты $x$, направленной вдоль вакуумной трубки от катода к аноду.

Используем одно из уравнений Максвелла в дифференциальной форме:
\[ \text{div} E = \frac{\rho}{\mathcal{E}_0} \]

где $\rho$ --- объёмная плотность заряда. Спроецируем это уравнение на ось $x$:
\begin{align} \label{Langmuir-ur1}
	\frac{\partial E}{\partial x} = - \frac1{\mathcal{E}_0} e n
\end{align}

Знак минус учитывает, что эмитируемые электроны имеют отрицательный заряд, $\mathcal{E}_0$ --- электрическая постоянная, $n$ --- концентрация электронов, $E$ --- напряжённость электрического поля. 

Работа по перемещению единичного точечного положительного заряда из одной точки поля в другую вдоль оси $x$ при условии, что точки расположены бесконечно близко друг к другу $x_2 - x_1 = dx$, равна $E_x dx$. Та же работа равна $\phi_1 - \phi_2 = -dU$. Приравняв оба выражения, можем записать:
\begin{align} \label{Langmuir-ur2}
	E_x = - \frac{dU}{dx}
\end{align}

Выразим напряжённость $E$ через потенциал $U$, с помощью формулы \eqref{Langmuir-ur2}. 
Заряд можно представить как количество элементарных зарядов(электронов) или заряд электрона на концентрацию в объёме:
\[ q = eN = enV \]
С другой стороны объёмная плотность заряда равна:
\[ \rho = \frac{dq}{dV} = \frac{d}{dV} enV = en  \]
Тогда получаем, что объёмная плотность тока равна:
\[ j = \rho u = enu \]
Концентрацию выразим через объёмную плотность тока, а скорость электронов - из закона сохранения энергии: $mu^2 / 2 = eU$: 
\begin{equation} \label{Langmuir-ur3}
	n = \frac{j}{eu} 
\end{equation}
\begin{equation} \label{Langmuir-ur4}
	u = \sqrt{ \frac{2eU}m } 
\end{equation}
Тогда уравнение \cref{Langmuir-ur1} с учётом \cref{Langmuir-ur2,Langmuir-ur3,Langmuir-ur4} принимает вид:
\[\frac{d}{dx} \frac{dU}{dx} = - \frac1{\mathcal{E}_0} e \frac{j}{e} \sqrt{\frac{m}{2eU}}\]
\begin{equation} \label{Langmuir-ur5}
	\frac{d^2U}{dx^2} = \frac{j}{\mathcal{E}_0} \sqrt{\frac{m}{2e}} U^{-\frac12}
\end{equation}

Решение дифференциального уравнения второго порядка \eqref{Langmuir-ur5} будем искать при граничных условиях $\lim\limits_{x \to 0}U = \lim\limits_{x \to 0}E = 0$. Если бы электрическое поле на границе катода было больше нуля, то все электроны, испускаемые катодом, увлекались бы этим полем к аноду, и термоэлектронный ток достигал бы насыщения при любых напряжениях на вакуумной трубке. Коэффициент перед $U^{-1/2}$ обозначим за $a$:
\begin{align} \label{Langmuir-ur6}
	\frac{d^2U}{dx^2} = a^2 U^{-\frac12}, \qquad a^2 = \frac{j}{\mathcal{E}_0} \sqrt{\frac{m}{2e}}
\end{align}
Введём замену:
\[ p = \frac{dU}{dx}, \qquad \frac{d^2U}{dx^2} = p \frac{dp}{dU} \]
Тогда, учитывая, что $x = 0:U_k = 0, dU/dx = 0$, получаем:
\[p \frac{dp}{dU} = a^2 U^{-\frac12}\]
\[p dp = a^2 U^{-\frac12} dU\]
\[\frac{p^2}{2} = 2 a^2 U^{\frac12}\]
\[p = 2 a U^{\frac14}\]
\[\frac{dU}{U^{1/4}} = 2 a dx\]
\[\frac43 U^{\frac34} = 2ax\]
\begin{equation} \label{Langmuir-ur7}
	a = \frac{2U^{3/4}}{3x}
\end{equation}
Подставим \eqref{Langmuir-ur7} в \eqref{Langmuir-ur6} и пологая, что $x = l$, где $l$ - расстояние между анодом и катодом, получаем:
\begin{align} \label{Langmuir-ur9}
	j = \frac{4\mathcal{E}_0}{9 l^2} \sqrt{\frac{2e}{m}} U^{\frac32}
\end{align}
Учитывая, что коэффициент перед $U$ -- константа, зависящая только от геометрических свойств прибора и фундаментальных постоянных(обозначим её $B$), то уравнение \eqref{Langmuir-ur9} можно переписать в виде: 
\begin{align} \label{Langmuir_fin}
	j = BU^{3/2}
\end{align}

\subsubsection{Связь динамической и статической крутизны}

По определению, коэффициенты крутизны -- статический $S_i$ и динамический $S_d$, а так же внутренние сопротивление триода $R_i$ запишутся как
\begin{align} \label{eq:link:1}
	\begin{aligned}
		S_d = \frac{d I_a}{d U_c} && R_i = \frac{d U_a}{знамеd I_a} && S_i = \frac{\partial I_a}{\partial U_c}
	\end{aligned}
\end{align}
Из закона Ома для цепи источника анода следует
\begin{equation} \label{eq:link:2}
	U_a = \Epsilon_a - I_a R_a
\end{equation}
Тогда дифференциал $U_a$ будет
\begin{equation} \label{eq:link:3}
	d U_a = -d I_a R_a
\end{equation}
Из формулы \cref{eq:link:2} можно выразить анодный ток
\begin{equation} \label{eq:link:4}
	I_a = \frac{\Epsilon_a}{R_a} - \frac{U_a}{R_a}
\end{equation}
И найти его полный дифференциал:
\begin{equation} \label{eq:link:5}
	d I_a = \frac{\partial I_a}{\partial U_c} d U_c + \frac{\partial I_a}{\partial U_a} d U_a = S_i\ dU_c + \frac{1}{R_i} d U_a
\end{equation}
Тогда после несложных преобразований
\begin{equation} \label{eq:link:6}
	\frac{d I_a}{d U_c } = S_i + \frac{1}{R_i} \frac{d U_a}{d U_c} = S_i - \frac{R_a}{R_i} \frac{d I_a}{d U_c} = S_i - \frac{R_a}{R_i} S_d
\end{equation}
И окончательный результат
\begin{equation} \label{eq:link:7}
	S_d \left[1 + \frac{R_a}{R_i}\right] = S_i
\end{equation}
\begin{equation} \label{eq:link:8}
	S_d = \frac{S_i}{1 + \frac{R_a}{R_i}}
\end{equation}

\subsection{Обработка и построение графиков}
\subsubsection{Аппроксимация теоретического графика крутизны $S$}

Из формулы закона трех вторых \cref{Langmuir_fin} мы имеем {\it степенную зависимость} данных. Для построения теоретического графика заменим переменные тока и напряжения на оси $x$ ($x = U_a$) и $y = I_a$:
\begin{equation} \label{eq:51}
	y = Bx^{\tfrc{3}{2}}
\end{equation}

Будем использовать метод наименьших квадратов для получения значения переменной $g$ (первеанс).
\[S = \sum_{i=1}^n \left(Bx_i^{\frac{3}{2}} - y_i\right)^2 \qquad S \to min\]

Для нахождения минимума функции суммы необходимо вычислить ее частную производную по $g$ и приравнять к нулю.
\[\frac{\partial S}{\partial B} = \sum_{i=1}^n 2\left(Bx_i^{\frac{3}{2}} - y_i\right)x_i^{\frac{3}{2}} = 2 \sum_{i=1}^n \left(B x_i^3 - y_i x_i^{\frac{3}{2}}\right) \qquad \frac{\partial S}{\partial B} = 0\]
\[B \sum_{i=1}^n x_i^3 = \sum_{i=1}^n y_i x_i^{\frac{3}{2}}\]
\begin{equation} \label{eq:52}
	\boxed{B = \frac{\sum_{i=1}^n y_i x_i^{\frac{3}{2}}}{\sum_{i=1}^n x_i^3}}
\end{equation}


%%%%%%%%%%%%%%%%%%%%%%%%%%%%%%%%%%%%%%%%%%%%%%%%%%%%%%%%%%%%
\subsection{Вычисление погрешностей}
\subsubsection{Приборная погрешность источника питания постоянного тока Б5-50}

Приборная погрешность установки выходного напряжения в режиме стабилизации напряжения не превышает:
\begin{align} \label{leq:error:1.1}
	\Delta U_\text{приб} = \pm (0,5\% U + 0,1\% U_\text{макс}),
\end{align}
где $U$ и $U_\text{макс}$ -- устанавливаемое и максимальное значение выходного напряжения прибора.

Из характеристик прибор мы знаем, что $U_\text{макс} = 299 \text{ В}$, поэтому:
\begin{align} \label{eq:error:1.2}
	\boxed{\Delta  U_\text{приб} = 0,005U + \frac{299}{1000} \text{ В}}
\end{align}

\subsubsection{Приборная погрешность GDM-8245 при измерении постоянного тока}

Приборная погрешность GDM-8245 при измерении постоянного тока:
\begin{align} \label{eq:error:2.1}
	\boxed{\Delta I_\text{приб} = \pm (0,002 \cdot I + 2 \cdot dI)}
\end{align}
где $I$ и $dI$ -- получаемый ток и цена деления измерения прибора.

\subsubsection{Косвенная погрешность крутизны S диода}

Выведем формулу косвенной погрешности крутизны диода из формулы \cref{eq:2}:
\[S = \frac{\Delta I_a}{\Delta U_a} = \frac{I_{a_2} - I_{a_1}}{U_{a_2} - U_{a_1}}\]  
\begin{align} \label{eq:error:3.1}
	\Delta S = \sqrt{\left( \frac{\partial S}{\partial I_{a_1}} \Delta I_{a_1}\right)^2 + \left( \frac{\partial S}{\partial I_{a_2}} \Delta I_{a_2}\right)^2 + \left( \frac{\partial S}{\partial U_{a_1}} \Delta U_{a_1}\right)^2 + \left( \frac{\partial S}{\partial U_{a_2}} \Delta U_{a_2}\right)^2}
\end{align}
\begin{multline} \label{eq:error:3.2}
	\Delta S = \sqrt{\frac{\Delta I_{a_1}^2}{(U_{a_2} - U_{a_1})^2} + \frac{\Delta I_{a_2}^2}{(U_{a_2} - U_{a_1})^2} + \frac{(I_{a_2} - I_{a_1})^2 }{(U_{a_2} - U_{a_1})^4} \Delta U_{a_1}^2 + \frac{(I_{a_2} - I_{a_1})^2 }{(U_{a_2} - U_{a_1})^4} \Delta U_{a_2}^2} = \\ = \boxed{\frac{\sqrt{(\Delta I_{a_1}^2 + \Delta I_{a_2}^2)(U_{a_2} - U_{a_1})^2 + (\Delta U_{a_1}^2 + \Delta U_{a_2}^2)(I_{a_2} - I_{a_1})^2}}{(U_{a_2} - U_{a_1})^2}}
\end{multline}

\subsubsection{Приборная погрешность генератора низкочастотных сигналов ГЗ-112}

Относительная приборная погрешность ГЗ-112 при создании сигнала $\nu$:
\begin{align} \label{eq:error:4.1}
	\Delta \nu_\text{пр} = \pm \left(2 + \frac{30}{\nu_\text{уст}}\right)\%
\end{align}
где $\nu_\text{уст}$ --- значение, установленное на шкале регулятора частоты. Распишем:
\begin{align} \label{eq:error:4.2}
	\boxed{\Delta \nu_\text{пр} = \pm \left(\frac{2\,\nu_\text{уст} + 30}{100} k\right)}
\end{align}
где $k$ --- значение множителя на генераторе.

\subsubsection{Вычисление косвенной погрешности коэффициента K}
Выведем формулу косвенной погрешности крутизны из формулы \cref{eq:pr:1}:
\[K = \frac{2 U_\text{вых}}{2 U_\text{вх}}\]
\begin{multline} \label{eq:error:5.1}
	\Delta K = \sqrt{\left( \frac{\partial K}{\partial 2 U_\text{вых}} \Delta 2U_\text{вых} \right)^2 + \left(\frac{\partial K}{\partial 2U_\text{вх}} \Delta 2U_\text{вх}\right)^2} =\\= \sqrt{\frac{\Delta (2U_\text{вых})^2}{(2U_\text{вх})^2} + \frac{(2U_\text{вых})^2 \Delta (2U_\text{вх})^2}{(2U_\text{вх})^4}}
\end{multline}

\subsubsection{Вычисление косвенной погрешности статической крутизны триода}
Выведем формулу косвенной погрешности статической крутизны из формулы \cref{eq:2.1}:
\begin{align} \label{eq:error:6.1}
	\Delta S = \sqrt{\left( \frac{\partial S}{\partial I_{a_1}} \Delta I_{a_1}\right)^2 + \left( \frac{\partial S}{\partial I_{a_2}} \Delta I_{a_2}\right)^2 + \left( \frac{\partial S}{\partial U_{c_1}} \Delta U_{c_1}\right)^2 + \left( \frac{\partial S}{\partial U_{c_2}} \Delta U_{c_2}\right)^2}
\end{align}
\begin{multline} \label{eq:error:6.2}
	\Delta S = \sqrt{\frac{\Delta I_{a_1}^2}{(U_{c_2} - U_{c_1})^2} + \frac{\Delta I_{a_2}^2}{(U_{c_2} - U_{c_1})^2} + \frac{(I_{a_2} - I_{a_1})^2 }{(U_{c_2} - U_{c_1})^4} \Delta U_{c_1}^2 + \frac{(I_{a_2} - I_{a_1})^2 }{(U_{c_2} - U_{c_1})^4} \Delta U_{c_2}^2} = \\ = \boxed{\frac{\sqrt{(\Delta I_{a_1}^2 + \Delta I_{a_2}^2)(U_{c_2} - U_{c_1})^2 + (\Delta U_{c_1}^2 + \Delta U_{c_2}^2)(I_{a_2} - I_{a_1})^2}}{(U_{c_2} - U_{c_1})^2}}
\end{multline}

