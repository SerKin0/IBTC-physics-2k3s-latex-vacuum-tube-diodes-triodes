\section{Теоретическая часть}

\subsection{Диод}

Конструктивно диод состоит из баллона (стеклянного, металлического или керамического), в котором создается вакуум $10^{-7}$ мм ртутного столба, и системы плоских или цилиндрических электродов: катода и анода.

Катод в простейшем случае представляет собой накаливаемую током вольфрамовую нить, которая при достаточно высокой температуре начинает испускать электроны (это явление получило название термоэлектронной эмиссией). Но такие прямоканальные катоды применяются очень редко; значительно большее распространение получили катоды с косвенным подогревом, где источником электронов служит эмиттер, 