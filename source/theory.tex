\section{Теоретическая часть}

\subsection{Диод}

Конструктивно диод состоит из баллона (стеклянного, металлического или керамического), в котором создается вакуум $\sim 10^{-7}$ мм ртутного столба, и системы плоских или цилиндрических электродов: катода и анода.

Катод в простейшем случае представляет собой накаливаемую током вольфрамовую нить, которая при достаточно высокой температуре начинает испускать электроны (это явление получило название \textit{термоэлектронной эмиссии}).

Если соединить анод с катодом через чувствительный гальванометр, то можно обнаружить в этой цепи анодный ток, величина определённая электрическая мощность. Электроны, излучаемые катодом, увеличивают свою кинетическую энергию за счет энергии электрического поля. При ударе об анод электрон отдаёт ему свою энергию. Энергия электронов выделяется на аноде в виде тепловой энергии и излучается им в окружающее пространство. С повышением анодного напряжения увеличивается количество достигших анода электронов, их скорость и кинетическая энергия, и, следовательно, возрастает и мощность, рассеиваемая анодом. Для каждого типа ламп существует максимальная допустимая величина этой мощности -- $P_{max}$. Превышение
ее при работе ламп может вывести последнюю из строя. Величину мощности, рассеиваемой анодом, можно подсчитать по формуле
\begin{equation} \label{eq:0}
	P_a = U_a I_a
\end{equation}

Уравнение $P_{a}^{max} = U_a I_a = const$ на плоскости $U_a$, $I_a$ изображается гиперболой, асимптотами которой являются координатные оси. На рис.2 эта кривая отмечена индексом "1". Чтобы лампа не вышла из строя, всегда надо следить за тем, чтобы при работе с ней не превышать максимально допустимой мощности, рассеиваемой анодом, т.е. работать только в области $P_a < P_{a}^{max}$. Для каждого типа ламп величина $P_{a}^{max}$ определяется конструкцией электродов и их геометрическими размерами. У очень мощных ламп для повышения этой мощности приходится применять меры принудительного охлаждения анода (водяное и воздушное охлаждение).

Функция, выражающая зависимость анодного тока диода от величины напряжения накала $I_a = f(U_n)$, называется температурной характеристикой диода. По ней можно судить об эмиссионной способности катода, о возникновении и исчезновении пространственного заряда между катодом и анодом лампы. Для каждого типа ламп существует нормальное напряжение накала, величину которого нельзя превышать.

Представление о работе диода можно составить и не зная его характеристик, если известны параметры диода: крутизна вольт-амперной характеристики 
\begin{equation} \label{eq:1}
	S = \frac{d I_a}{d U_a},
\end{equation}
измеряемая обычно в $\tfrac{\text{мА}}{\text{В}}$ или обратная ей величина -- внутренние сопротивление $R_i = \frac{1}{S}$. Т.к. характеристика диода существенно нелинейна, то это дифференциальные параметры зависят от значений $I_a$ и $U_a$, т.е. $S = f(U_a)$. На практике приближенное значение крутизны можно определить по вольт-амперной характеристике. Взяв небольшое приращение анодного напряжения $\Delta U_a$ так, чтобы в этих пределах участок анодной характеристики можно считать линейным, по характеристике определим приращение тока $\Delta I_a$. И тогда
\begin{equation} \label{eq:2}
	S \approx \frac{\Delta I_a}{\Delta U_a}
\end{equation}



%%%%%%%%%%%%%%%%%%%%%%%%%%%%%%%%%%%%%%%%%%%%%%%%%%%%%%%%%%%%
\subsection{Вывод закона 3/2 Ленгмюра}
Определим зависимость анодного тока от анодного напряжения в диоде, образованном двумя плоскими, безграничными, параллельными друг другу пластинами. В этом случае можно пренебречь краевым эффектом и считать поле между анодом и катодом однородным. Примем следующие допущения. Пусть у поверхности катода $x = 0: U_k = 0$, , начальная скорость электронов $u_0 = 0$.

Если предположить, что электроды вакуумной трубки плоские, а температура катода постоянная, то потенциал электрического поля будет зависеть только от одной координаты $x$, направленной вдоль вакуумной трубки от катода к аноду.

Используем одно из уравнений Максвелла в дифференциальной форме:
\[ \text{div} E = \frac{\rho}{\mathcal{E}_0} \]

где $\rho$ - объёмная плотность заряда. Спроецируем это уравнение на ось $x$:

\begin{align} \label{Langmuir-ur1}
	\frac{\partial E}{\partial x} = - \frac1{\mathcal{E}_0} e n
\end{align}

Знак минус учитывает, что эмиттируемые электроны имеют отрицательный заряд, $\mathcal{E}_0$ - электрическая постоянная, $n$ - концентрация электронов, $E$ - напряжённость электрического поля. 

Работа по перемещению единичного точечного положительного заряда из одной точки поля в другую вдоль оси $x$ при условии, что точки расположены бесконечно близко друг к другу $x_2 - x_1 = dx$, равна $E_x dx$. Та же работа равна $\phi_1 - \phi_2 = -dU$. Приравняв оба выражения, можем записать:

\begin{align} \label{Langmuir-ur2}
	E_x = - \frac{dU}{dx}
\end{align}

Выразим напряжённость $E$ через потенциал $U$, с помощью формулы \eqref{Langmuir-ur2}. 
Заряд можно представить как количество элементарных зарядов(электронов) или заряд электрона на концентрацию в объёме:
\[ q = eN = enV \]

С другой стороны объёмная плотность заряда равна:
\[ \rho = \frac{dq}{dV} = \frac{d}{dV} enV = en  \]
Тогда получаем, что объёмная плотность тока равна:
\[ j = \rho u = enu \]

Концентрацию выразим через объёмную плотность тока, а скорость электронов - из закона сохранения энергии: $mu^2 / 2 = eU$: 
\begin{align} \label{Langmuir-ur3}
	n = \frac{j}{eu} 
\end{align}
\begin{align} \label{Langmuir-ur4}
	u = \sqrt{ \frac{2eU}m } 
\end{align}

Тогда уравнение \eqref{Langmuir-ur1} с учётом \eqref{Langmuir-ur2}, \eqref{Langmuir-ur3} и \eqref{Langmuir-ur4} принимает вид:
\begin{align*}
	\frac{d}{dx} \frac{dU}{dx} = - \frac1{\mathcal{E}_0} e \frac{j}{e} \sqrt{\frac{m}{2eU}}
\end{align*}

\begin{align} \label{Langmuir-ur5}
	\frac{d^2U}{dx^2} = \frac{j}{\mathcal{E}_0} \sqrt{\frac{m}{2e}} U^{-\frac12}
\end{align}

Решение дифференциального уравнения второго порядка \eqref{Langmuir-ur5} будем искать при граничных условиях $\lim\limits_{x \to 0}U = \lim\limits_{x \to 0}E = 0$. Если бы электрическое поле на границе катода было больше нуля, то все электроны, испускаемые катодом, увлекались бы этим полем к аноду, и термоэлектронный ток достигал бы насыщения при любых напряжениях на вакуумной трубке. Коэффициент перед $U^{-1/2}$ обозначим за $a$:
\begin{align} \label{Langmuir-ur6}
	\frac{d^2U}{dx^2} = a^2 U^{-\frac12}, \qquad a^2 = \frac{j}{\mathcal{E}_0} \sqrt{\frac{m}{2e}}
\end{align}

Введём замену:
\[ p = \frac{dU}{dx}, \qquad \frac{d^2U}{dx^2} = p \frac{dp}{dU} \]
Тогда, учитывая, что $x = 0:U_k = 0, dU/dx = 0$, получаем:
\begin{align*}
	p \frac{dp}{dU} = a^2 U^{-\frac12} \\
	p dp = a^2 U^{-\frac12} dU \\
	\frac{p^2}{2} = 2 a^2 U^{\frac12} \\
	p = 2 a U^{\frac14}
\end{align*}
\begin{align*}
	\frac{dU}{U^{1/4}} = 2 a dx \\
	\frac43 U^{\frac34} = 2ax
\end{align*}
\begin{align} \label{Langmuir-ur7}
	a = \frac{2U^{3/4}}{3x}
\end{align}
Подставим \eqref{Langmuir-ur7} в \eqref{Langmuir-ur6} и пологая, что $x = l$, где $l$ - расстояние между анодом и катодом, получаем:
\begin{align} \label{Langmuir-ur9}
	j = \frac{4\mathcal{E}_0}{9 l^2} \sqrt{\frac{2e}{m}} U^{\frac32}
\end{align}
Учитывая, что коэффициент перед $U$ - константа, зависящая только от геометрических свойств прибора и фундаментальных постоянных(обозначим её $B$), то уравнение \eqref{Langmuir-ur9} можно переписать в виде: 
\begin{align} \label{Langmuir_fin}
	j = BU^{3/2}
\end{align}


%%%%%%%%%%%%%%%%%%%%%%%%%%%%%%%%%%%%%%%%%%%%%%%%%%%%%%%%%%%%
\subsection{Обработка и построение графиков}
\subsubsection{Аппроксимация теоретического графика крутизны $S$}

Из формулы закона трех вторых мы имеем {\it степенную зависимость} данных. Для построения теоретического графика заменим переменные тока и напряжения на оси $x$ ($x = U_a$) и $y = I_a$:
\begin{equation} \label{eq:51}
	y = Bx^{\tfrc{3}{2}}
\end{equation}

Будем использовать метод наименьших квадратов для получения значения переменной $g$ (первеанс).
\[S = \sum_{i=1}^n \left(Bx_i^{\frac{3}{2}} - y_i\right)^2 \qquad S \to min\]

Для нахождения минимума функции суммы необходимо вычислить ее частную производную по $g$ и приравнять к нулю.
\[\frac{\partial S}{\partial B} = \sum_{i=1}^n 2\left(Bx_i^{\frac{3}{2}} - y_i\right)x_i^{\frac{3}{2}} = 2 \sum_{i=1}^n \left(B x_i^3 - y_i x_i^{\frac{3}{2}}\right) \qquad \frac{\partial S}{\partial B} = 0\]
\[B \sum_{i=1}^n x_i^3 = \sum_{i=1}^n y_i x_i^{\frac{3}{2}}\]
\begin{equation} \label{eq:52}
	\boxed{B = \frac{\sum_{i=1}^n y_i x_i^{\frac{3}{2}}}{\sum_{i=1}^n x_i^3}}
\end{equation}


%%%%%%%%%%%%%%%%%%%%%%%%%%%%%%%%%%%%%%%%%%%%%%%%%%%%%%%%%%%%
\subsection{Вычисление погрешностей}
\subsubsection{Приборная погрешность источника питания постоянного тока Б5-50}

Приборная погрешность установки выходного напряжения в режиме стабилизации напряжения не превышает:
\begin{align} \label{leq:error:1.1}
	\Delta U_\text{приб} = \pm (0,5\% U + 0,1\% U_\text{макс}),
\end{align}
где $U$ и $U_\text{макс}$ -- устанавливаемое и максимальное значение выходного напряжения прибора.

Из характеристик прибор мы знаем, что $U_\text{макс} = 299 \text{ В}$, поэтому:
\begin{align} \label{eq:error:1.2}
	\boxed{\Delta  U_\text{приб} = 0,005U + \frac{299}{1000} \text{ В}}
\end{align}

\subsubsection{Приборная погрешность GDM-8245 при измерении постоянного тока}

Приборная погрешность GDM-8245 при измерении постоянного тока:
\begin{align} \label{eq:error:2.1}
	\boxed{\Delta I_\text{приб} = \pm (0,002 \cdot I + 2 \cdot dI)}
\end{align}
где $I$ и $dI$ -- получаемый ток и цена деления измерения прибора.

\subsubsection{Косвенная погрешность крутизны S}

Выведем формулу косвенной погрешности крутизны из формулы \cref{eq:2}:
\[S = \frac{\Delta I_a}{\Delta U_a} = \frac{I_{a_2} - I_{a_1}}{U_{a_2} - U_{a_1}}\]  
\begin{align} \label{eq:error:3.1}
	\Delta S = \sqrt{\left( \frac{\partial S}{\partial I_{a_1}} \Delta I_{a_1}\right)^2 + \left( \frac{\partial S}{\partial I_{a_2}} \Delta I_{a_2}\right)^2 + \left( \frac{\partial S}{\partial U_{a_1}} \Delta U_{a_1}\right)^2 + \left( \frac{\partial S}{\partial U_{a_2}} \Delta U_{a_2}\right)^2}
\end{align}
\begin{multline} \label{eq:error:3.2}
	\Delta S =\\= \sqrt{\frac{\Delta I_{a_1}^2}{(U_{a_2} - U_{a_1})^2} + \frac{\Delta I_{a_2}^2}{(U_{a_2} - U_{a_1})^2} + \frac{(I_{a_2} - I_{a_1})^2 }{(U_{a_2} - U_{a_1})^4} \Delta U_{a_1}^2 + \frac{(I_{a_2} - I_{a_1})^2 }{(U_{a_2} - U_{a_1})^4} \Delta U_{a_2}^2} = \\ = \boxed{\frac{\sqrt{(\Delta I_{a_1}^2 + \Delta I_{a_2}^2)(U_{a_2} - U_{a_1})^2 + (\Delta U_{a_1}^2 + \Delta U_{a_2}^2)(I_{a_2} - I_{a_1})^2}}{(U_{a_2} - U_{a_1})^2}}
\end{multline}

\subsubsection{Приборная погрешность генератора низкочастотных сигналов ГЗ-112}

Относительная приборная погрешность ГЗ-112 при создании сигнала $\nu$:
\begin{align} \label{eq:error:4.1}
	\Delta \nu_\text{пр} = \pm (2 + \frac{30}{\nu_\text{уст}})\%
\end{align}
где $\nu_\text{уст}$ -- значение, установленное на шкале регулятора частоты. Распишем:
\begin{align} \label{eq:error:4.2}
	\boxed{\Delta \nu_\text{пр} = \pm \left(\frac{2\,\nu_\text{уст} + 30}{100} k\right)}
\end{align}
где $k$ -- значение множителя на генераторе.

\subsubsection{Вычисление косвенной погрешности коэффициента K}
Выведем формулу косвенной погрешности крутизны из формулы \cref{eq:pr:1}:
\[K = \frac{2 U_\text{вых}}{2 U_\text{вх}}\]
\begin{equation} \label{eq:error:5.1}
	\Delta K = \sqrt{\left( \frac{\partial K}{\partial 2 U_\text{вых}} \Delta 2U_\text{вых} \right)^2 + \left(\frac{\partial K}{\partial 2U_\text{вх}} \Delta 2U_\text{вх}\right)^2}
\end{equation}
\begin{align} \label{eq:error:5.2}
	\Delta K = \sqrt{\frac{\Delta (2U_\text{вых})^2}{(2U_\text{вх})^2} + \frac{(2U_\text{вых})^2 \Delta (2U_\text{вх})^2}{(2U_\text{вх})^4}}
\end{align}