\section{Контрольные вопросы}

\begin{librarybox}
	1. Объясните качественно ход анодной характеристики лампы, включенной диодом.
\end{librarybox}

\begin{librarybox}
	2. Пренебрегая начальной скоростью вылета электронов из катода и считая поле между анодом и катодом однородным, а анодное напряжение заданным, найти скорость электронов у поверхности анода и время пролета электроном расстояния между катодом и анодом.
\end{librarybox}


\begin{librarybox}
	3. Каков будет анодный ток триода, если его сетку оставить никуда не подключенной?
\end{librarybox}


\begin{librarybox}
	4. Каков будет характер движения электронов, если на сетку подать большой положительный потенциал, а на анод небольшой отрицательный по отношению к катоду?
\end{librarybox}


\begin{librarybox}
	5. В чем отличие внутреннего сопротивления триода от его сопротивления постоянному току?
\end{librarybox}


\begin{librarybox}
	6. Чем определяются частотные свойства триода?
\end{librarybox}